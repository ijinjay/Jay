%!TEX program = xelatex
%!TEX encoding = UTF-8 unicode
% 某个公司需要在三个月内完成一个财务系统取代原有的财务系统,原有的财务系统能满足日常工作需求,但是性能存在较大问题,因此希望可以开发出一套基于更安全的新平台的财务系统,并拥有一群开发过旧软件系统的人员,希望文档尽量完备。

\documentclass[UTF8,nofonts]{ctexart}
\setCJKmainfont[BoldFont=STSong,ItalicFont=STKaiti]{STSong}
\setCJKsansfont[BoldFont=STHeiti]{STXihei}
\setCJKmonofont{STFangsong}
\setcounter{secnumdepth}{5}
\usepackage[hmarginratio=1:1]{geometry}
\usepackage{array,float,graphicx}
\usepackage{pstricks-add}
%表格style
\newcolumntype{L}[1]{>{\vspace{0.5em}\begin{minipage}{#1}\raggedright\let\newline\\
\arraybackslash\hspace{0pt}}m{#1}<{\end{minipage}\vspace{0.5em}}}
\newcolumntype{R}[1]{>{\vspace{0.5em}\begin{minipage}{#1}\raggedleft\let\newline\\
\arraybackslash\hspace{0pt}}m{#1}<{\end{minipage}\vspace{0.5em}}}
\newcolumntype{C}[1]{>{\vspace{0.5em}\begin{minipage}{#1}\centering\let\newline\\
\arraybackslash\hspace{0pt}}m{#1}<{\end{minipage}\vspace{0.5em}}}
%section style
\makeatletter
\renewcommand{\section}{\@startsection{section}{1}{0mm}
  {-\baselineskip}{0.5\baselineskip}{\fontsize{16pt}{16pt}\bf\leftline}}
\makeatother
%引用style
\makeatletter
\def\@cite#1#2{\textsuperscript{[{#1\if@tempswa , #2\fi}]}}
\makeatother
\pagestyle{empty}
%封面
\title{\Huge 软件过程简介}
\author{\large	
  软件工程1102班 靳杰 U201112375}
\date{\Large\today}
\begin{document}
\maketitle
\onecolumn{
\tableofcontents}
\newpage
\section{问题描述}
某个公司需要在三个月内完成一个财务系统取代原有的财务系统,原有的财务系统能满足日常工作需求,但是性能存在较大问题,因此希望可以开发出一套基于更安全的新平台的财务系统,并拥有一群开发过旧软件系统的人员,希望文档尽量完备。
\section{问题分析}
该公司此次软件项目在原有的旧财务系统需求上加入安全因素和性能需求,故软件系统需求较明确。又因为公司希望开发文档尽量完备,考虑多种软件开发模型,其中瀑布模型以文档为驱动,适用于需求明确的项目,所以初步考虑使用瀑布模型。但是项目时间要求三个月,时间较为紧迫。不过,由于新系统用于取代旧系统,财务系统具有需求明确、一次交付等特征,瀑布模型适用。综上,该软件项目采用瀑布模型。
\section{软件过程} % (fold)
\label{sec:软件过程实现_}
\subsection{基本过程} % (fold)
\label{sub:基本过程_}
\subsubsection{需求概述} % (fold)
\emph{原始需求}

公司财务系统主要包括:
\begin{itemize}
\setlength{\itemsep}{0pt}
\setlength{\parskip}{0pt}
\setlength{\parsep}{0pt}
\item 固定资产管理子系统
\item 流动资产管理子系统
\item 成本管理子系统
\item 企业收入管理子系统
\item 财务收支管理子系统
\end{itemize}

公司现有的财务系统能满足日常工作需求,但是性能存在较大问题。并且此次系统开发会基于更安全的新平台进行开发。故新系统还需具有如下特征:
\begin{itemize}
\setlength{\itemsep}{0pt}
\setlength{\parskip}{0pt}
\setlength{\parsep}{0pt}
	\item 公司职员可以根据权限浏览个人的财务管理系统,不会发生越界访问。
	\item 公司职员财务管理系统能够接受全公司所有人员的共同访问而不发生崩溃,响应时间小于3秒。
\end{itemize}

\emph{需求分析与规格说明}

财务系统是一个内部联机系统,所有的财务数据都保存在一个中心数据库中。每天由财务处统计更新数据,每月进行一次财务分析,员工可以每天登陆查看个人财务信息。

财务系统的需求规格说明书如表1所示:\\
\noindent\rule[0.25\baselineskip]{\textwidth}{2pt}

% subsubsection 原始需求_ (end)
% subsection 基本过程_ (end)
% section 软件过程实现_ (end)
\end{document}